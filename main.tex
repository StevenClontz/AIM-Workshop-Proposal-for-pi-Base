\documentclass{amsart}
\usepackage[utf8]{inputenc}

\usepackage{xcolor}
\usepackage[colorlinks,urlcolor=blue,citecolor=blue]{hyperref}

\newcommand{\TODO}[1]{\textcolor{red}{(TODO: #1)}}

\subjclass[2010]{54A05}

\title{AIM Workshop Proposal for pi-Base}
%\author{Steven Clontz}
\date{\today}

\begin{document}

\maketitle

%\abstract{asdf}

In Summer 1967, Lynn Steen and J. Arthur Seebach, Jr. gathered together
an REU of five students to canvas the field of topology and create
a compilation of topological spaces, properties, and theorems.
The result of this work was \textit{Counterexamples in Topology} 
\cite{MR1382863}, which went on to inspire several other texts
with the same goal of modeling the objects, properties, and theorems of
their fields \cite{MR0491272,MR1113487,MR930671,MR1256489}.

The pi-Base (\url{http://pi-base.org}) is a community-curated
database of topological spaces, properties, and theorems, 
inspired by \textit{Counterexamples} and founded in 2014 by
workshop proposer James Dabbs. More
powerful than a wiki, the pi-Base software features automated
deduction and self-completion based upon user contributions;
for example,
\href{https://topology.pi-base.org/theorems/I000013}{Theorem 13}
asserts that all compact spaces are paracompact,
and
\href{https://topology.pi-base.org/theorems/I000170}{Theorem 170}
asserts that all paracompact Hausdorff spaces are fully \(T_4\).
As a result, a space such as
\href{https://topology.pi-base.org/spaces/S000095}{Example 95}
which is asserted to be compact and Hausdorff is automatically tagged
as fully \(T_4\). 

The database backing the pi-Base is fully open-source and available
at \url{https://github.com/pi-base/data}.
For most of its existence, contributions were
unreferreed, but due to recent work by workshop proposer Steven Clontz
and his student,
all assertions in the pi-Base now reference peer-reviewed literature.
New contributions can be made easily through the pi-Base interface,
which issues pull requests via GitHub for review before merging 
these contributions into the master branch.

Over the years, the pi-Base has grown in popularity,
as witnessed by its reference on Math.StackExchange 
(e.g. \cite{157918}), mathematical blogs (e.g. \cite{lamb_2018}),
research presentations (e.g. \cite{clontzPiBase}),
and lists of online tools for research mathematicians 
(e.g. \cite{scheepersPiBase}). Part of this interest is sparked
by the fact that the pi-Base platform is not
hard-coded to support topology; indeed,
any mathematical category that can be sufficiently modeled by
its objects, properties, and theorems could have its own
pi-Base. 
\TODO{Add info on who has contacted Dabbs about pi-Base for their field.}

In addition to its value for experienced researchers, the applicants
feel that a vibrant pi-Base will have great benefits for 
undergraduate and graduate students in the each field it models.
Indeed, we anticipate that much of the
data entry from the existing peer-reviewed literature
will rely on contributions from such students, who will need to
peruse this content nonetheless. In addition, a completed pi-Base
can reveal gaps in the peer-reviewed literature, generating truly
open questions that will often be appropriate for student research.
This potential is witnessed by its use in the mathematics classroom
by Austin Mohr of Nebraska Wesleyan \cite{mohrPiBase}, 
and by its usage in a Summer 2019 REU run by Jocelyn Bell of
Hobart and William Smith \TODO{cite} in the vein of the original
\textit{Counterexamples} project mentioned previously.

Due to these developments, the applicants believe the time is ripe
for a workshop that will enable the wider adoption of the
pi-Base as a first-class tool for cutting-edge research in
general and set-theoretic topology, as well as establish
a second instance of the pi-Base platform
modeling another category of mathematics
\TODO{which?}. We know of no other workshops in this general area.

\section{Workshop Plan}

The \href{https://www.aimath.org/research/aimstyle.html}{AIM format}
will be adhered to for this workshop. The first few talks will be
designed to orient participants to the current progress of the pi-Base
project, including the challenges that will be addressed during the workshop.
Likely topics include the following.

\begin{itemize}
    \item The history of cataloging mathematical counterexamples 
    \item Using and contributing to the pi-Base 
    \item An overview of the technology powering the pi-Base
    \item A crash-course in \TODO{outside field} and its parallels to
      general topology
    \item Set-theoretic challenges to modeling topology
\end{itemize}

\noindent Later talks will be chosen based upon the progress made during the
workshop.

We anticipate the following working groups.

\begin{itemize}
    \item Content curation: researchers will identify the
      most useful areas for expansion of the topology database. 
    \item Student contributions: based upon the literature identified
      by researchers, students will contribute content to the database,
      refereed by researchers. 
    \item Feature requests: which missing features of the pi-Base
      are most crucial for expanding its utility to researchers of
      topology and other
      fields? How can these features be modeled abstractly as to
      be of use to various fields?
    \item Reviewing/contribution workflow: how can the pi-Base
      community facilitate efficient adoption of the platform
      by more contributors and referees?
    \item Expansion: begin the process of establishing a pi-Base
      to model \TODO{external field}.
    \item Software development: based upon the live usage of the
      pi-Base by students and researchers in various fields, 
      developers (with previous experience with mathematical research) 
      can efficiently
      implement bugfixes and feature requests.
\end{itemize}

\section{List of Organizers}

Steven Clontz will serve as the lead organizer. Dr. Clontz is an Assistant
Professor of Mathematics at the University of South Alabama who specializes
in the area of topological game theory and serves as the mathematical editor
for the pi-Base. 
Clontz has extensive experience with software development, having
previously collaborated with Dabbs on an online CRM platform for connecting
collegiate music organizations with their members, recruits, and alumni;
he currently maintains SbgRails, an open-source platform 
for facilitating mastery grading in mathematics classrooms. He has
previously participated in an AIM workshop dedicated to the Curated Courses
project. His focus is the development of the mathematical content
of the pi-Base platform, including its expansion into other fields.

James Dabbs will serve on the organizing team as the founder of the pi-Base
and a professional software developer at \TODO{where} 
with experience with mathematical
research from his master's thesis in general and algebraic topology.
He will lead the continued development of the pi-Base software to
serve its mathematical audiences.

Lynne Yengulalp is an Associate Professor of Mathematics at the
University of Dayton. She will serve on the organizing team as an accomplished researcher
of general topology and set theory, representing the majority of
the topological research community without software development experience.
Among her goals is ensuring
that the user experience of the pi-Base is appropriately designed for
use by the research community, both for browsing and contributing.

\TODO{It may make sense to invite an outside researcher as an organizer.}

\section{Potential Participants}

In the following list of potential participants, (F) is used to denote
female participants, (M) for underrepresented minorities, (U) for
faculty at undergraduate institutions, (J) for junior researchers
(students and untenured faculty), (G) for international participants, 
and (I) for those working
outside academia. Organizers are listed in bold.

\begin{itemize}
    \item Lead Organizer
    \begin{itemize}
        \item \textbf{Steven Clontz} (J), University of South Alabama, Topology
    \end{itemize}
    \item Topology Researchers
    \begin{itemize}
        \item \textbf{Lynne Yengulalp} (F), University of Dayton, Topology
        \item Jocelyn Bell (F,U,J), Hobart and William Smith Colleges, Topology
        \item Will Brian (J), UNC Charlotte, Topology
        \item Ziqin Feng, Auburn University, Topology
        \item Jared Holshouser (J), University of South Alabama, Set Theory and Topology
        \item Paul Szeptycki (G), York University, Set Theory and Topology
        \item \TODO{five more} 
    \end{itemize}
    \item Undergraduate and Graduate Students
    \begin{itemize}
        \item Six students identified by the topology research team, with a
          focus on recruiting women and underrepresented minorities. (F,M,J)
    \end{itemize}
    \item \TODO{External field, 4 participants}
    \item Software Developers 
    \begin{itemize}
        \item \textbf{James Dabbs} (I), \TODO{TODO}, Topology
        \item Zachary Sarver (I), \TODO{TODO}, Algebra
        \item \TODO{1 or 2 more}
    \end{itemize}
\end{itemize}

\TODO{This is 25-26 participants, workshop supports up to 28 so we have flexibility}

\bibliography{mybib}{}
\bibliographystyle{plain}

\end{document}

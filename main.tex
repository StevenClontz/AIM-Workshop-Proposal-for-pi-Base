\documentclass{amsart}
\usepackage[utf8]{inputenc}

\usepackage{xcolor}
\usepackage[colorlinks,urlcolor=blue,citecolor=blue]{hyperref}

\newcommand{\TODO}[1]{\textcolor{red}{(TODO: #1)}}

%\subjclass[2010]{54A05}

\title{pi-Base Database of Topological Spaces}
%\author{Steven Clontz}
\date{\today}

\begin{document}

\maketitle

%\abstract{asdf}

\section{Historical Context}

In the summer of 1967, Lynn Steen and J. Arthur Seebach, Jr. gathered together
an REU of five students to canvas the field of topology and create
a compilation of topological spaces, properties, and theorems, with particular
attention to how these spaces serve as counterexamples to the converses of
known theorems.
The result of this work was \textit{Counterexamples in Topology} 
\cite{MR1382863}, whose first edition published in 1970 
went on to inspire several other \textit{Counterexamples}
texts of the same spirit in other fields, such as graph theory, differential
equations, probability, and real analysis
\cite{MR0491272,MR1113487,MR930671,MR1256489}.

In the spirit of these works,
we propose a web application that can power modern versions
of such \textit{Counterexamples} texts, extending far beyond the scope
of any printed book. This project, which we call the pi-Base, is based 
on general principles that apply to multiple areas of mathematics.
We see the pi-Base as the next logical step beyond the textbooks
and wikis that currently fill this need for cataloging mathematical
objects studied in the literature.

\section{About the pi-Base}

The pi-Base (\url{http://pi-base.org}) is a community-curated
database of topological spaces, properties, and theorems, 
inspired by \textit{Counterexamples} and founded in 2014 by
workshop proposer James Dabbs. More
powerful than a wiki, the pi-Base software features automated
deduction and self-completion based upon user contributions;
for example,
\href{https://topology.pi-base.org/theorems/I000013}{Theorem 13}
asserts that all compact spaces are paracompact (noting
that \href{https://topology.pi-base.org/spaces/S000025}{Example 25}
is a counterexample to the contrary), and
\href{https://topology.pi-base.org/theorems/I000170}{Theorem 170}
asserts that all paracompact Hausdorff spaces are fully \(T_4\).
As a result, a space such as
\href{https://topology.pi-base.org/spaces/S000095}{Example 95}
which is asserted to be compact and Hausdorff is automatically tagged
as fully \(T_4\). A video introduction to the pi-Base presented at the
52nd Spring Conference on Topology and Dynamics may be
watched at \url{https://youtu.be/iuNZWllVDKg?t=3909}.

The database backing the pi-Base is fully open-source and available
at \url{https://github.com/pi-base/data}.
For most of its existence, contributions were
unreferreed, but due to recent work by workshop proposer Steven Clontz
and his student,
all assertions in the pi-Base now reference peer-reviewed literature.
New contributions are made through the pi-Base web application, 
which provides a user-friendly interface abstracting the internal
mechanics of the Git repository modeling the database.
Rather than using the Git revision control system directly, users
are provided with a wiki-like interface allowing them to make their
contributions via a web browser, and then click a button which automates 
the process of issuing a ``pull request''
on GitHub.com allowing for these changes to be reviewed by the community
before being merged with the other accepted contributions.

Over the years, the pi-Base has grown in popularity,
as witnessed by its reference on Math.StackExchange 
(e.g. \cite{157918}), mathematical blogs (e.g. \cite{lamb_2018}),
research presentations (e.g. \cite{clontzPiBase}),
and lists of online tools for research mathematicians 
(e.g. \cite{scheepersPiBase}). However, the community that has
actively developed its content and software is small in comparison
to the number of users that visit the pi-Base; more work needs to be
done to convert more of these visitors into contributors. 
%We believe part of this interest is sparked
%by the fact that the pi-Base platform is not
%hard-coded to support topology; indeed,
%any mathematical category that can be sufficiently modeled by
%its objects, properties, and theorems could have its own
%pi-Base. 

\section{Broader Impact}

While an important goal of the pi-Base is to serve as a convenient
tool for experienced researchers, 
a vibrant pi-Base will also have great benefits for 
undergraduate and graduate students in the each field it models.
Indeed, we anticipate that much of the
data entry from the existing peer-reviewed literature
will rely on contributions from such students.
In addition, a completed pi-Base
will reveal gaps in the peer-reviewed literature, generating truly
open questions that will often be appropriate for student research;
for example, witnessing the absence of a counterexample in the literature
for the converse of a theorem cataloged in the pi-Base.
This potential is witnessed by the use of the pi-Base in an undergraduate
topology course instructed by Austin Mohr of Nebraska Wesleyan \cite{mohrPiBase}, 
and by its usage in an upcoming Summer 2019 REU run by Jocelyn Bell of
Hobart and William Smith in the vein of the original
\textit{Counterexamples in Topology} project mentioned previously.

To this end, the proposed workshop will directly involve students,
with a particular effort made to target underrepresented groups.

\section{Workshop Plan}

The time is ripe for a workshop that will enable the wider adoption of the
pi-Base as a first-class tool for cutting-edge research in
general topology, particularly set-theoretic topology and continuum
theory. Beyond just populating the database with more results from
these fields, the proposed workshop will develop the software powering
the pi-Base with the ultimate aim of making it compatible with cataloging
the other important counterexamples found in other fields of mathematics.
We know of no other workshops in this general area.


The \href{https://www.aimath.org/research/aimstyle.html}{AIM format}
will be adhered to for this workshop. The applicants aim to achieve
the following goals.

\begin{itemize}
    \item Expand the community of active pi-Base contributors by training
    pi-Base users (researchers and students) to integrate the pi-Base
    into their workflow outside the context of the workshop.
    \item Populate the database with more results from the literature
    based upon the areas of active research that would best benefit
    from it, with particular attention to areas that could also foster
    undergraduate and graduate research.
    \item Develop the software powering the pi-Base, taking advantage
    of the unique opportunity to do so alongside active users of
    the platform.
    \item Identify necessary features to extend the pi-Base platform
    to model other areas of mathematics.
\end{itemize}

%The first few talks will be
%designed to orient participants to the current progress of the pi-Base
%project, including the challenges that will be addressed during the workshop.
%Likely topics include the following.

%\begin{itemize}
%    \item The history of cataloging mathematical counterexamples 
%    \item Using and contributing to the pi-Base 
%    \item An overview of the technology powering the pi-Base
%    \item A crash-course in \TODO{outside field} and its parallels to
%      general topology
%    \item Set-theoretic challenges to modeling topology
%\end{itemize}

%\noindent Later talks will be chosen based upon the progress made during the
%workshop.

%We anticipate the following working groups.

%\begin{itemize}
%    \item Content curation: researchers will identify the
%      most useful areas for expansion of the topology database. 
%    \item Student contributions: based upon the literature identified
%      by researchers, students will contribute content to the database,
%      refereed by researchers. 
%    \item Feature requests: which missing features of the pi-Base
%      are most crucial for expanding its utility to researchers of
%      topology and other
%      fields? How can these features be modeled abstractly as to
%      be of use to various fields?
%    \item Reviewing/contribution workflow: how can the pi-Base
%      community facilitate efficient adoption of the platform
%      by more contributors and referees?
%    \item Expansion: begin the process of establishing a pi-Base
%      to model \TODO{external field}.
%    \item Software development: based upon the live usage of the
%      pi-Base by students and researchers in various fields, 
%      developers (with previous experience with mathematical research) 
%      can efficiently
%      implement bugfixes and feature requests.
%\end{itemize}

\section{List of Organizers}

Steven Clontz will serve as the lead organizer. Clontz is an Assistant
Professor of Mathematics at the University of South Alabama who specializes
in the area of topological game theory and serves as the mathematical editor
for the pi-Base. 
Clontz has extensive experience with software development, having
previously collaborated with Dabbs on an online CRM platform for connecting
collegiate music organizations with their members, recruits, and alumni;
he currently develops SbgRails, an open-source platform 
for facilitating mastery grading in mathematics classrooms. He has
previously participated in an AIM workshop dedicated to the Curated Courses
project and will participate in AIM's Summer 2019 PreTeXT workshop. 
His focus is the development of the mathematical content
of the pi-Base platform, including its expansion into other fields.

James Dabbs will serve on the organizing team as the founder of the pi-Base
and a professional software developer at Procore Technologies 
with experience with mathematical
research from his master's thesis in general and algebraic topology.
Dabbs will lead the continued development of the pi-Base software to
serve its mathematical audiences along with other experienced software
developers with mathematical backgrounds.

Lynne Yengulalp is an Associate Professor of Mathematics at the
University of Dayton. Yengulalp will serve on the organizing team as an accomplished researcher
of general topology, representing the end-users of the pi-Base outside
its software development team.
Among her goals is ensuring that the pi-Base is appropriately designed for
use by the general research community, both for browsing and contributing.

%\TODO{It may make sense to invite an outside researcher as an organizer.}
\bibliography{mybib}{}
\bibliographystyle{plain}

\newpage

%
%\section{Potential Participants}

%In the following list of potential participants, (F) is used to denote
%female participants, (M) for underrepresented minorities, (U) for
%faculty at undergraduate institutions, (J) for junior researchers
%(students and untenured faculty), 
%and (I) for those working
%outside academia. Organizers are listed in bold.

%\begin{itemize}
%%    \item Lead Organizer
%%    \begin{itemize}
%%    \end{itemize}
%    \item Mathematical Researchers
%    \begin{itemize}
%        \item Lori Alvin (F,J), Furman University, Topology
%        \item Jocelyn Bell (F,J,U), Hobart and William Smith Colleges, Topology
%        \item Will Brian (J), UNC Charlotte, Topology
%        \item Nathan Carlson, California Lutheran, Topology
%        \item Christopher Caruvana (J,M), IU Kokomo, Set Theory and Topology
%        \item \textbf{Steven Clontz} (J), University of South Alabama, Topology
%        \item Ziqin Feng, Auburn University, Topology
%        \item Jared Holshouser (J), University of South Alabama, Set Theory and Topology
%        \item David Milovich (I), Welkin Sciences, Topology
%        \item Lon Mitchell, American Mathematical Society, Mathematical Reviews
%        \item Ted Porter, Murray State, Topology
%        \item Marion Scheepers, Boise State University, Set Theory
%        \item Paul Szeptycki (G), York University, Set Theory and Topology
%        \item \textbf{Lynne Yengulalp} (F), University of Dayton, Topology
%    \end{itemize}
%    \item Undergraduate and Graduate Students
%    \begin{itemize}
%        \item Several students studying under participating faculty will participate,
%          with a focus on recruiting women and underrepresented minorities. (F,J,M)
%    \end{itemize}
%%    \item \TODO{External field, 4 participants}
%    \item Software Developers 
%    \begin{itemize}
%        \item \textbf{James Dabbs} (I), Procore Technologies, Topology
%%        \item Daniel Brice (I), Lumi, Algebra
%        \item Evin Sellin (I), Procore Technologies, Computer Science
%        \item Zachary Sarver (I), Vectorworks, Algebra
%    \end{itemize}
%\end{itemize}

%
\end{document}
